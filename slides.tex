%\documentclass[11pt,handout]{beamer}
\documentclass{beamer}

\usepackage[utf8]{inputenc}
\usepackage{listings}

\usepackage{graphicx}
\usepackage{hyperref}

\hypersetup{%
pdftitle={A brief introduction to Dart},
pdfauthor={David Stutz},
pdfkeywords={Dart},
bookmarksnumbered,
pdfstartview={FitH},
colorlinks,
citecolor=black,
filecolor=black,
linkcolor=black,
urlcolor=black,
breaklinks=true,
}

% For colors
\usepackage{color}

% Define colors used for listings
\definecolor{dkgreen}{rgb}{0,0.6,0}
\definecolor{gray}{rgb}{0.5,0.5,0.5}
\definecolor{mauve}{rgb}{0.58,0,0.82}

% Configuration for listings
\lstset{%
numbers=left,
basicstyle=\small,
numbers=left,
numberstyle=\tiny,
numbersep=5pt,
tabsize=2,
flexiblecolumns=true,
keywordstyle=\color{blue},
commentstyle=\color{dkgreen}, 
stringstyle=\color{mauve},
numberstyle=\tiny\color{gray},
language=Java,
breaklines=true,
breakatwhitespace=true,
morekeywords={*,num,String,var,library,get,set} ,
}

\usetheme{Goettingen}
\usecolortheme{whale}

\setbeamertemplate{footline}[frame number]
\setbeamertemplate{sections/subsections in toc}[square]

\begin{document}

\title{Dart}
\subtitle{Eine kurze Einführung}
\author{David Stutz}
\institute{Lehrstuhl für Datenmanagement und -exploration \\RWTH Aachen}
\date{Proseminar SS 2012}
\subject{Dart}

% \AtBeginSection[]
% {
%   \begin{frame}
%     \frametitle{Table of Contents}
%     \tableofcontents[currentsection]
%   \end{frame}
% }

\frame{\titlepage}

\begin{frame}
\frametitle{Überblick}
\tableofcontents
\end{frame}

\section{Einführung}
\begin{frame}
\frametitle{Einführung}

\begin{figure}[htb]
	\centering
 		\includegraphics[width=0.4\textwidth]{images/dart-logo.png}
\end{figure}\footnote{Bild: \url{http://www.dartlang.org/}}

\begin{itemize}
\item Klassen-basierte Programmiersprache für Webanwendungen
\item Entwickelt von Google
\pause
\item Vorgestellt auf der GOTO Konferenz in Aarhus am 12. Oktober 2011
\pause
\item Zurzeit: ``technical preview''
\pause
\item Diese Präsentation: Dart für clientseitige Programmierung
\end{itemize}
\end{frame}

\section{Motivation}
\begin{frame}
\frametitle{Motivation}
\begin{itemize}
\item ``Building delightful applications on the web today is far too difficult.''\footnote{\url{https://gist.github.com/1208618}} -- Mark S. Miller
\pause
\item Komplexe Anwendungen benötigen modularen und strukturierten Code
\pause
\item Kaum Möglichkeiten homogene Systeme zu entwickeln
\pause
\begin{itemize}
\item Google Web Toolkit\\\url{https://developers.google.com/web-toolkit/}
\item Node.js\\\url{http://nodejs.org/}
\end{itemize}
\pause
\item Clientseitig vorallem Schwierigkeiten mit Javascript
\end{itemize}
\end{frame}

\section{Werkzeuge}
\begin{frame}
\frametitle{Werkzeuge}

Google stellt hilfreiche Werkzeuge zur Entwicklung bereit:
\begin{itemize}
\item Dart--zu--Javascript Übersetzer
\pause
\item Dart Editor -- auf Eclipse basierter Editor
\pause
\item ``Dartium'' -- auf Chromium basierter Browser mit integrierter Dart VM
\end{itemize}
\pause
Downloads unter \url{http://dartlang.org}.
\end{frame}

\section{Demonstration}
\begin{frame}
\frametitle{Demonstration}
\begin{center}
\huge{Demonstration}
\end{center}
\end{frame}

%\subsection{Main-Methode}
%\begin{frame}[fragile]
%\frametitle{Main-Methode}
%
%Der Einstiegspunkt in Dart ist die \lstinline|main()| Methode:
%
%\begin{lstlisting}
%void main() {
%  new Contacts().run();
%}
%\end{lstlisting}
%\pause
%\begin{lstlisting}
%class Contacts {
%  Contacts() {
%    // Konstruktor der Klasse.
%  }
%  void run() {
%    // Oeffentliche Methode ohne Rueckgabewert.
%    // Wird spaeter Initialisierung durchfuehren.
%  }
%}
%\end{lstlisting}

%\end{frame}

\section{Dart}
\subsection{Klassen}
\begin{frame}[fragile]
\frametitle{Klassen}

Dart unterstützt:
\begin{itemize}
\item Klassen-basierte Objektorientierung
\pause
\item Einfache Vererbung
\pause
\item Abstrakte Klassen
\pause
\item Interfaces
\end{itemize}
\pause
Außerdem:
\begin{itemize}
\item Verzicht auf \lstinline|public|/\lstinline|private|
\item Getter/Setter mittels \lstinline|get|/\lstinline|set| deklarieren
\item Benannte Konstruktoren
\end{itemize}
%\pause
%\begin{lstlisting}
%class Model_Contact {
%  
%  // Attribute sind automatisch privat auf Grund des '_'-Praefix.
%  num _id;
%  String _first_name;
%  String _last_name;
%  String _phone;
%  String _email;
%  // ...
%\end{lstlisting}
\end{frame}

\subsection{Optionale Typen}
\begin{frame}[fragile]
\frametitle{Optionale Typen}

Eingebaute Typen:
\begin{itemize}
\item Zahlen -- \lstinline|num|
\item Zeichenketten -- \lstinline|String|
\item ``Collections'' -- Listen, Assoziative Felder...
\end{itemize}
\pause
Aber:
\begin{itemize}
\item Typen nicht verpflichtend
\item Variablen alternativ mit \lstinline|var| deklarierbar
%\begin{lstlisting}
%// Eine Zeichenkette deklarieren:
%String typedString = 'Wird dieser Variable ein anderer Variablentyp zugewiesen erkennt der "Type Checker" es.';
%var untypedString = 'Prinzipiell sind all Variablen ungetypt.';
%\end{lstlisting}
%\pause
%\item Der ``Type Checker'' erkennt Typfehler, diese behindern allerdings nicht die Ausführung
\end{itemize}
\end{frame}

\subsection{DOM}
\begin{frame}[fragile]
\frametitle{DOM -- Document Object Model}
%Dart stellt eine Art ``Collection Framework'' bereit:
%\begin{itemize}
%\item Assoziative Felder (Maps)
%\pause
%\item Listen
%\end{itemize}
%\pause
%Jeweils mit verschiedenen Implementierungen.
%\vskip 1em
%\pause
DOM Manipulation -- Zugriff auf HTML:
\begin{itemize}
\item Neue DOM Elemente erstellen:
\begin{itemize}
\item \lstinline|Element.tag('tag')|
\item \lstinline|Element.html('<html>...</html>')|
\end{itemize}
\pause
\item Auf Attribute direkt mittels \lstinline|element.attributes| zugreifen
\pause
\item Knoten von DOM Elementen mittels \lstinline|element.nodes| ansprechen
\end{itemize}
%\pause
%\begin{lstlisting}
%class Model {
%  static List<Model_Contact> contacts; // Wird die Liste der Kontakte sein.
%  static void add(Model_Contact contact) {
%    if (Model.contacts == null) {
%      Model.contacts = []; // Initialisiere Liste
%    }
%    Model.contacts.add(contact); // Ein neues Element hinzufuegen.
%    contact.id = Model.contacts.indexOf(contact); // Index eines Elements herausfinden.
%  }
%  // ...
%\end{lstlisting}
\end{frame}

%\subsection{DOM}
%\begin{frame}[fragile]
%\frametitle{DOM}
%
%\begin{lstlisting}
%Element row = new Element.tag('tr'); // Neues Zeilen-ELement.
%Element submit = new Element.tag('td'); // Neues Zellen-Element.
%submit.attributes['colspan'] = '2';
%submit.innerHTML = '<button type="submit" id="submit">Submit</button>';
%row.nodes.add(submit); // Zelle zur Zeile hinzufuegen.
%this.element.nodes.add(row); // Zeile zur Tabelle Hinzufuegen.
%\end{lstlisting}
%\pause
%\begin{itemize}
%\item Neue DOM Elemente mittels \lstinline|Element.tag('tag')| erstellen
%\pause
%\item Auf Attribute direkt mittels \lstinline|attributes| zugreifen
%\pause
%\item Knoten von DOM Elementen mittels \lstinline|nodes| ansprechen
%\end{itemize}
%\end{frame}

\subsection{Modularität}
\begin{frame}[fragile]
\frametitle{Modularität}

Anwendungen lassen sich modular aufbauen:
\begin{itemize}
\item Eigene Bibliotheken erstellen:
\begin{lstlisting}
// In View.dart eine Bibliothek definieren:
#library('view');
\end{lstlisting}
\pause
\item Bibliotheken einbinden:
\begin{lstlisting}
// Dart's HTML Bibliothek einbinden:
#import('dart:html');
// Eigene Bibliothek einbinden:
#import('View.dart');
\end{lstlisting}
\end{itemize}

\end{frame}

\section{Evaluation}
\begin{frame}
\frametitle{Evaluation}
\begin{center}
\begin{tabular}{ p{0.45\textwidth} || p{0.45\textwidth}}
Dart & Javascript\\ [0.4em]
\hline \\ [0.2em]
Klassen-basiert, Interfaces, Vererbung & Prototypen-basiert, komplizierte Vererbung\\ [0.4em]
Optionale Typen & erschwerte Typüberprüfung \newline \begin{itemize} \item ``undefined'' \item ``falsify''\end{itemize}\\ [0.4em]
Bibliotheken & Keine Bibliotheken \newline -- Globaler Namespace\\ [0.4em]
DOM Manipulation mittels CSS Selektoren & externe Bibliotheken (z.B. JQuery\footnote{\url{http://jquery.com/}})\\ [0.4em]
\end{tabular}
\end{center}
%Dart's Stärken:
%\begin{itemize}
%\item Bekannte Syntax
%\pause
%\item Klassen-basierte Objektorientierung
%\pause
%\item Optionale Typen
%\pause
%\item Nicht gezeigte Features wie:
%\begin{itemize}
%\item Isolates
%\end{itemize}
%\end{itemize}
\end{frame}

\begin{frame}
\frametitle{Evaluation (II)}

%Von Javascript gelernt:
%\begin{itemize}
%\item Klassen-basierte Objektorientierung
%\pause
%\item Erleichterte Typüberprüfung (optionale Typen)
%\pause
%\item Eingebaute Möglichkeiten zum Umgang mit Bibliotheken
%\pause
%\item Einfache DOM Manipulation
%\end{itemize}
%\pause

Probleme für den clientseitigen Einsatz:
\begin{itemize}
\item Übersetzung zu Javascript vor Ausführung
\pause
\item Nativer Browser-Support --  ``as of March 2012, Microsoft Internet Explorer, Mozilla Firefox, Opera Software's Opera browser, and Apple Safari do not have plans to implement support for Dart''\footnote{\url{http://en.wikipedia.org/wiki/Dart_(programming_language)}}
\end{itemize}
\end{frame}

\section{Quellen}
\begin{frame}
\frametitle{Quellen}
\begin{thebibliography}{10}    
  \beamertemplatebookbibitems
  \bibitem{Dartlang}
    Dartlang.org
    \newblock {\url{http://dartlang.org}}
  \bibitem{Wikipedia}
    Dart (programming language)
    \newblock {\url{http://en.wikipedia.org/wiki/Dart_(programming_language)}}
  \bibitem{t3n}
    Dart: 10 Punkte, in denen es JavaScript übertrifft
    \newblock {\url{http://t3n.de/news/dart-10-punkte-denen-javascript-358345/}}
  \beamertemplatearticlebibitems
  \end{thebibliography}
\end{frame}

\begin{frame}
\frametitle{Ende}
\begin{center}
\huge{Vielen Dank für die Aufmerksamkeit!}\\
\vskip 1em
\LARGE{Fragen?}
\end{center}
\end{frame}

\section{Zusatz}
\subsection{Typ-Überprüfung}
\begin{frame}[fragile]
\frametitle{Typ-Überprüfung}
\begin{figure}[!h]
	\begin{minipage}{0.49\textwidth}
		\centering
		\begin{lstlisting}[lang=Javascript,basicstyle=\scriptsize]
			var nullVal = null;
			if (!nullVal) {
			  // 'null' wird als 'false' behandelt...
			}
		\end{lstlisting}
	\end{minipage}
	\begin{minipage}{0.49\textwidth}
		\centering
		\begin{lstlisting}[basicstyle=\scriptsize]
			var nullVal = null;
			if (nullVal == null) {
			  // Dart kennt lediglich das 'wahre' false!
			}
		\end{lstlisting}
	\end{minipage}
\end{figure}
\pause
\begin{figure}[!h]
	\begin{minipage}{0.49\textwidth}
		\centering
		\begin{lstlisting}[lang=Javascript,basicstyle=\scriptsize]
			var emptyString = '';
			if (!emptyString) {
			  // leere Zeichenketten werden als 'false' behandelt...
			}
		\end{lstlisting}
	\end{minipage}
	\begin{minipage}{0.49\textwidth}
		\centering
		\begin{lstlisting}[basicstyle=\scriptsize]
			var emptyString = '';
			if (emptyString.isEmpty()) {
			  // ...
			}
		\end{lstlisting}
	\end{minipage}
\end{figure}
\pause
\begin{figure}[!h]
	\begin{minipage}{0.49\textwidth}
		\centering
		\begin{lstlisting}[lang=Javascript,basicstyle=\scriptsize]
			var undefinedVal;
			if (!undefinedVal) {
			  // 'undefined' wird auch als 'false' behandelt...
			}
		\end{lstlisting}
	\end{minipage}
	\begin{minipage}{0.49\textwidth}
		\centering
		\begin{lstlisting}[basicstyle=\scriptsize]
			// Dart kennt kein 'undefined'!
		\end{lstlisting}
	\end{minipage}
\end{figure}
\end{frame}

\subsection{Optionale Parameter}
\begin{frame}[fragile]
\frametitle{Optionale Parameter}
\begin{figure}[!h]
	\begin{minipage}{0.49\textwidth}
		\centering
		\begin{lstlisting}[lang=Javascript,basicstyle=\scriptsize]
			function foo(x, y, z) {
			  return z;
			}

			foo(1); // Wird 'undefined' zurueckgeben...
		\end{lstlisting}
	\end{minipage}
	\begin{minipage}{0.49\textwidth}
		\centering
		\begin{lstlisting}[basicstyle=\scriptsize]
			// Rueckgabewert in der Deklaration anzugeben ist optional:
			foo(x, y, z) {
			  return z;
			}

			foo(1); // NoSuchMethodException!

			// y und z sind optionale Parameter:
			bar(x, [y, z]) {
			  return z;
			}

			bar(1); // Wird 'null' zurueckgeben!
		\end{lstlisting}
	\end{minipage}
\end{figure}
\end{frame}

\subsection{Klassen}
\begin{frame}[fragile]
\frametitle{Klassen}
\begin{figure}[!h]
	\begin{minipage}{0.49\textwidth}
		\centering
		\begin{lstlisting}[lang=Javascript,basicstyle=\scriptsize]
			// Ein Objekt erstellen:
			function Car(brand) {
			  this.brand = brand;
			}

			// Dem Prototypen eine Methode hinzufuegen:
			Car.prototype.alertBrand = function() {
			  alert(this.brand);
			}
		\end{lstlisting}
	\end{minipage}
	\begin{minipage}{0.49\textwidth}
		\centering
		\begin{lstlisting}[basicstyle=\scriptsize]
			class Car {
			
			  // Durch '_'-Praefix automatisch privat:
			  var _brand;
			
			  // Konsturktor wird _brand automatisch zuweisen:
			  Car(this._brand);

			  alertBrand() {
			    window.alert(this._brand);
			  }
			}
		\end{lstlisting}
	\end{minipage}
\end{figure}
\end{frame}

%\subsection{Getters und Setters}
%\begin{frame}[fragile]
%\frametitle{Getters und Setters}
%
%Dart unterstützt extra \lstinline|set| and \lstinline|get| keywords:
%\begin{lstlisting}
%// 'get' deklariert einen Getter.
%String get first_name() => this._first_name;
%// 'set' deklariert einen Setter.
%set first_name(String first_name) => this._first_name = first_name;
%\end{lstlisting}
%\pause
%Das Attribut kann nun wie ein öffentliches Attribut behandelt werden:
%\begin{lstlisting}
%contact.first_name = 'David';
%print(contact.first_name);
%\end{lstlisting}
%\pause
%Außerdem:
%\begin{itemize}
%\item Abkürzungen für Methodendeklarationen
%\end{itemize}
%\end{frame}
\end{document}